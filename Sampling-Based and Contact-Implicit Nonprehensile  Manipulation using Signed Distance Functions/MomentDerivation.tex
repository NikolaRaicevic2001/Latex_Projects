\documentclass[conference]{IEEEtran}

%%%%%%%%%%%%%%%%%%%%%%%%%%%%%
%%%%%%%% PACKAGES %%%%%%%%%%%
%%%%%%%%%%%%%%%%%%%%%%%%%%%%%
\usepackage{times}
\usepackage[numbers]{natbib}
\usepackage{multicol}
\usepackage[bookmarks=true]{hyperref}
\usepackage{amsmath}
\usepackage{amssymb}
\usepackage{graphicx}
\usepackage{caption}
\usepackage{stfloats} 

\begin{document}

\title{Planar Inertia via SDF and a Periodic Spline Boundary}
\maketitle
\IEEEpeerreviewmaketitle

% Setup and Notation
\subsection{Setup and Notation}
\noindent Let $\Omega\subset\mathbb{R}^2$ denote the planar cross-section of a rigid slider in its body frame, with constant thickness $h>0$ (out of plane) and uniform density $\rho>0$. We take a signed distance field (SDF) $\phi:\mathbb{R}^2\!\to\mathbb{R}$ with the convention
\[
\Omega \;=\; \{\,x\in\mathbb{R}^2:\ \phi(x)<0\,\}.
\]
The total mass, centroid, and planar (polar) second moment about the centroid are
\begin{align}
A &= \int_{\Omega} dA, \qquad
m =\int_{\Omega}\rho\,h\,dA,\\
c \;&=\; \frac{1}{A}\int_{\Omega} x\,dA \;\in\mathbb{R}^2,\\
J_o \;&=\; \int_{\Omega} \|x\|^2\,dA, \qquad
J_c \;=\; J_o - A\,\|c\|^2,\\
I_z \;&=\; \rho\,h\,J_c.
\end{align}
When useful, a smooth indicator $H_\varepsilon(\phi)$ (e.g. $H_\varepsilon(\phi)=\tfrac12(1-\tanh(\phi/\varepsilon))$) may replace $\mathbf{1}_{\{\phi<0\}}$ so that
\begin{align}
A \;=\; \int_{\mathbb{R}^2} H_\varepsilon(\phi)\,dA,\quad
c \;=\; \frac{1}{A}\int_{\mathbb{R}^2} x\,H_\varepsilon(\phi)\,dA,\quad \\
J_o \;=\; \int_{\mathbb{R}^2} \|x\|^2\,H_\varepsilon(\phi)\,dA
\end{align}

% Boundary Extraction from the SDF
\subsection{Boundary Extraction from the SDF}
\paragraph{Stable interior center}
Define an interior seed as the SDF minimizer
\begin{equation}
c_0 \;\in\; \arg\min_{x\in\mathbb{R}^2}\ \phi(x),
\end{equation}
e.g. by gradient descent with backtracking:
\begin{equation}
x_{k+1} \;=\; x_k \;-\; \alpha_k\,\nabla\phi(x_k),
\; \text{stop when } \|\nabla\phi(x_k)\| \le \tau
\end{equation}

\paragraph{Polar directions and seeding.}
Pick $n$ angles $\theta_i=\tfrac{2\pi i}{n}$ with unit directions $u_i=[\cos\theta_i,\ \sin\theta_i]^\top$. Choose a radius $R>0$ enclosing the object and set
\begin{equation}
x_i^{(0)} \;=\; c_0 \;+\; R\,u_i,\qquad i=0,\dots,n-1.
\end{equation}

\paragraph{Projection to the zero level set.}
Project each seed to the boundary $\{\phi=0\}$ by Newton projection:
\begin{equation}
x_i^{(\ell+1)} = x_i^{(\ell)} - \frac{\phi\!\left(x_i^{(\ell)}\right)}{\|\nabla\phi\!\left(x_i^{(\ell)}\right)\|^2+\varepsilon}\,\nabla\phi\!\left(x_i^{(\ell)}\right),
\quad \text{stop when } \big|\phi(x_i^{(\ell)})\big| \le \tau
\label{eq:newton-proj}
\end{equation}
(If $\phi$ is a perfect SDF with $\|\nabla\phi\|\equiv 1$, \eqref{eq:newton-proj} reduces to $x^+ = x - \phi(x)\,\nabla\phi(x)$.) Denote the converged boundary samples by $b_i$.

% Periodic B-Spline Enclosure
\subsection{Periodic B-Spline Enclosure}
\noindent Fit a closed (periodic) spline $C:[0,1]\to\mathbb{R}^2$, e.g. a periodic cubic B-spline,
\begin{equation}
C(t) \;=\; \sum_{j=0}^{M-1} P_j\,N_{j,p}(t),\qquad p=3,\quad t\in[0,1),
\end{equation}
with periodic basis $\{N_{j,p}\}$ and control points $\{P_j\}$ (indices modulo $M$). One may obtain $C$ by periodic least-squares fitting of $\{b_i\}$ (optionally with Tikhonov smoothing on control-point second differences). Finally, ensure $C$ is oriented counter-clockwise (CCW).

% Green's Theorem
\subsection{Area, Centroid, and Inertia from the Spline (Green's Theorem)}
\noindent Let $C(t)=(x(t),y(t))$ be one CCW traversal of the boundary with $x'=\tfrac{dx}{dt},\ y'=\tfrac{dy}{dt}$. Then
\begin{align}
A \;&=\; \frac{1}{2}\int_0^1 \big( x\,y' - y\,x' \big)\,dt, \label{eq:area}\\[2pt]
c_x \;&=\; \frac{1}{6A}\int_0^1 \big( x^2\,y' - x\,y\,x' \big)\,dt, \qquad
c_y \;=\; \frac{1}{6A}\int_0^1 \big( y^2\,x' - x\,y\,y' \big)\,dt, \label{eq:centroid}\\[2pt]
J_o \;&=\; \frac{1}{2}\int_0^1 \big(x^2+y^2\big)\,\big( x\,y' - y\,x' \big)\,dt, \label{eq:polar-origin}\\[2pt]
J_c \;&=\; J_o \;-\; A\,\big(c_x^2+c_y^2\big), \qquad
I_z \;=\; \rho\,h\,J_c. \label{eq:Iz-final}
\end{align}
For piecewise-polynomial $C$, evaluate the integrals span-wise with Gaussian quadrature.

% Quasi-Static Dynamics
\subsection{Use in Quasi-Static Dynamics}
\noindent With $I_z$ from \eqref{eq:Iz-final}, the quasi-static resistive rotational balance writes
\begin{equation}
M_{\mathrm{res}} \;=\; \mu\,I_z\,\dot{\theta},\qquad
\dot{\theta} \;=\; \frac{1}{\mu\,I_z}\,M_{\mathrm{push}}
\end{equation}
The translational counterpart is $F_{\mathrm{res}}=\mu\,m\,g\,\dot{p}$ and $\dot{p}=\tfrac{1}{\mu\,m\,g} F_{\mathrm{push}}$.

\end{document}


